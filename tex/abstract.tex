%
%  Abstract
%

\begin{abstract}
\addcontentsline{toc}{chapter}{Abstract}
%todo: max 350 words

Chronic wounds represent a large and growing disease burden worldwide. Infection, biofilm formation, and associated pathological inflammation are some of the leading impediments to healing, suggesting an important role for the microbiome of these wounds. Studies of the bacterial fraction suggest that community composition, and its temporal variance, may be associated with healing outcomes, yet the forces that drive these dynamics are not well understood. The viral fraction of the microbiome, called the virome, may be a major contributing factor; other human viromes are dominated by bacteriophages, which not only infect and lyse bacteria, but can have profound impacts on host functionality. Despite its potential, the wound virome has not yet been described, largely due to the challenges associated with preparing and analyzing low-biomass clinical samples like those obtained from skin and wounds. 

To facilitate the study of skin and wound viromes, we developed an improved sample processing method for obtaining both viral-enriched and bacterial DNA from a single swab sample, resulting in higher yields and viral purity when compared to traditional methods. We then applied the improved swab processing protocol in a small-cohort metagenomic survey of the skin and chronic wound microbiomes of 20 volunteers at an outpatient wound care clinic. We report taxonomic composition and diversity of the bacterial and viral fractions, and their associations to clinical features. We find that bacterial oxygen requirements are associated with healing outcomes, as are specific bacteriophages and the auxiliary genes they carry. Additionally, we assessed how the microbiome is impacted by sharp debridement, a standard-of-care procedure that physically removes necrotic tissue. We found no significant differences between microbiomes before and immediately after debridement, confirming that the primary benefits of the treatment are longitudinal and derived from repeated procedures. This work establishes novel methodology for studying the wound microbiome and virome, confirms previous findings in the field, presents the first chronic wound virome findings, and identifies correlations and associations to healing outcomes that causation studies may investigate in the future.

%\abstractsignature
\end{abstract}


